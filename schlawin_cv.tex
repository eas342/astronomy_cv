\documentclass[11pt, oneside]{article}   	% use "amsart" instead of "article" for AMSLaTeX format
\usepackage[margin=1in]{geometry}                		% See geometry.pdf to learn the layout options. There are lots.
\geometry{letterpaper}                   		% ... or a4paper or a5paper or ... 
%\geometry{landscape}                		% Activate for rotated page geometry
%\usepackage[parfill]{parskip}    		% Activate to begin paragraphs with an empty line rather than an indent
\usepackage{graphicx}				% Use pdf, png, jpg, or eps§ with pdflatex; use eps in DVI mode
								% TeX will automatically convert eps --> pdf in pdflatex		
\usepackage{amssymb}
\usepackage{booktabs}
\usepackage[colorlinks=true,urlcolor=blue]{hyperref}
\usepackage{enumitem}
%\usepackage{bibunits}
\usepackage[numbers]{natbib}
\usepackage{xstring}

%\defaultbibliography{this_biblio}
%\defaultbibliographystyle{ieetr}

\setlength{\bibsep}{0pt plus 0.3ex}

\newcommand\aj{{AJ}}%        % Astronomical Journal 
\newcommand\araa{{ARA\&A}}%  % Annual Review of Astron and Astrophys 
\newcommand\apj{{ApJ}}%    % Astrophysical Journal ++
\newcommand\apjl{{ApJL}}     % Astrophysical Journal, Letters 
\newcommand\apjs{{ApJS}}%    % Astrophysical Journal, Supplement 
\newcommand\ao{{ApOpt}}%   % Applied Optics ++
\newcommand\apss{{Ap\&SS}}%  % Astrophysics and Space Science 
\newcommand\aap{{A\&A}}%     % Astronomy and Astrophysics 
\newcommand\aapr{{A\&A~Rv}}%  % Astronomy and Astrophysics Reviews 
\newcommand\aaps{{A\&AS}}%    % Astronomy and Astrophysics, Supplement 
\newcommand\azh{{AZh}}%       % Astronomicheskii Zhurnal 
\newcommand\baas{{BAAS}}%     % Bulletin of the AAS 
\newcommand\icarus{{Icarus}}% % Icarus
\newcommand\jrasc{{JRASC}}%   % Journal of the RAS of Canada 
\newcommand\memras{{MmRAS}}%  % Memoirs of the RAS 
\newcommand\mnras{{MNRAS}}%   % Monthly Notices of the RAS 
\newcommand\pra{{PhRvA}}% % Physical Review A: General Physics ++
\newcommand\prb{{PhRvB}}% % Physical Review B: Solid State ++
\newcommand\prc{{PhRvC}}% % Physical Review C ++
\newcommand\prd{{PhRvD}}% % Physical Review D ++
\newcommand\pre{{PhRvE}}% % Physical Review E ++
\newcommand\prl{{PhRvL}}% % Physical Review Letters 
\newcommand\pasp{{PASP}}%     % Publications of the ASP 
\newcommand\pasj{{PASJ}}%     % Publications of the ASJ 
\newcommand\qjras{{QJRAS}}%   % Quarterly Journal of the RAS 
\newcommand\skytel{{S\&T}}%   % Sky and Telescope 
\newcommand\solphys{{SoPh}}% % Solar Physics 
\newcommand\sovast{{Soviet~Ast.}}% % Soviet Astronomy 
\newcommand\ssr{{SSRv}}% % Space Science Reviews 
\newcommand\zap{{ZA}}%       % Zeitschrift fuer Astrophysik 
\newcommand\nat{{Nature}}%  % Nature 
\newcommand\iaucirc{{IAUC}}% % IAU Cirulars 
\newcommand\aplett{{Astrophys.~Lett.}}%  % Astrophysics Letters 
\newcommand\apspr{{Astrophys.~Space~Phys.~Res.}}% % Astrophysics Space Physics Research 
\newcommand\bain{{BAN}}% % Bulletin Astronomical Institute of the Netherlands 
\newcommand\fcp{{FCPh}}%   % Fundamental Cosmic Physics 
\newcommand\gca{{GeoCoA}}% % Geochimica Cosmochimica Acta 
\newcommand\grl{{Geophys.~Res.~Lett.}}%  % Geophysics Research Letters 
\newcommand\jcp{{JChPh}}%     % Journal of Chemical Physics 
\newcommand\jgr{{J.~Geophys.~Res.}}%     % Journal of Geophysics Research 
\newcommand\jqsrt{{JQSRT}}%   % Journal of Quantitiative Spectroscopy and Radiative Trasfer 
\newcommand\memsai{{MmSAI}}% % Mem. Societa Astronomica Italiana 
\newcommand\nphysa{{NuPhA}}%     % Nuclear Physics A 
\newcommand\physrep{{PhR}}%       % Physics Reports 
\newcommand\physscr{{PhyS}}%        % Physica Scripta 
\newcommand\planss{{Planet.~Space~Sci.}}%  % Planetary Space Science 
\newcommand\procspie{{Proc.~SPIE}}%      % Proceedings of the SPIE 

\newcommand\actaa{{AcA}}%  % Acta Astronomica
\newcommand\caa{{ChA\&A}}%  % Chinese Astronomy and Astrophysics
\newcommand\cjaa{{ChJA\&A}}%  % Chinese Journal of Astronomy and Astrophysics
\newcommand\jcap{{JCAP}}%  % Journal of Cosmology and Astroparticle Physics
\newcommand\na{{NewA}}%  % New Astronomy
\newcommand\nar{{NewAR}}%  % New Astronomy Review
\newcommand\pasa{{PASA}}%  % Publications of the Astron. Soc. of Australia
\newcommand\rmxaa{{RMxAA}}%  % Revista Mexicana de Astronomia y Astrofisica

\newcommand{\rsquo}{'}

\def\FormatName#1{%
  \def\myname{{Schlawin}, E.}%
  \edef\name{#1}%
  \ifx\name\myname
    \textbf{#1}%
  \else
    #1%
  \fi
}

%\def\FormatName#1{%
%  \IfSubStr{#1}{"{Schlawin}, E."}{\textbf{#1}}{#1}%
%}

%SetFonts

%SetFonts


\title{Everett Schlawin}


\date{}							% Activate to display a given date or no date

\begin{document}
\maketitle
\vspace{-0.85in}
\begin{centering}
\rule{6in}{0.03in}
Postdoctoral Researcher

268 Steward Observatory\\
933 N Cherry Ave, Tucson AZ 85721, USA

github: \href{https://github.com/eas342}{eas342}\\
eas342@email.arizona.edu\\
\url{http://mips.as.arizona.edu/~schlawin}

\end{centering}

\subsection*{Education}
\begin{itemize}[noitemsep]
	\item 2015: PhD, Cornell University in Ithaca, NY
		\begin{itemize}[noitemsep]
		\item Major: Astronomy, Minor: Earth and Atmospheric Sciences
		\item Thesis: Observations of Disintegrating, Evaporating and Hot Planet Atmospheres with Transmission Spectra
		\item Advisor: Professor Terry Herter
		\end{itemize}
	\item 2012: Masters, Cornell University in Ithaca, NY
		\begin{itemize}[noitemsep]
		\item Major: Astronomy, Minor: Earth and Atmospheric Sciences
		\end{itemize}
	\item 2009: Bachelors of Arts, Oberlin College in Oberlin, OH
		\begin{itemize}[noitemsep]
		\item Major: Physics w/ Astrophysics concentration
		\item Phi Beta Kappa, High Honors
		\end{itemize}
\end{itemize}

\subsection*{Academic Appointments}
\begin{itemize}[noitemsep]
	\item 2015-2018: Postdoctoral Researcher at the University of Arizona
\end{itemize}

\vspace{-0.2in}

\subsection*{Research Interests}
\begin{itemize}[noitemsep]
	\item Exoplanet and brown dwarf atmospheres
	\item Disintegrating planet systems
	\item JWST NIRCam and MIRI
	\item Near infrared Instrumentation
\end{itemize}
\vspace{-0.2in}

\subsection*{Publications}
\nocite{schlawin2018kic1255Normal}
\nocite{schlawin2018JWSTforecasts}
\nocite{schlawin2014}
\nocite{schlawin2017dhs}
\nocite{schlawin2017bdVar}
\nocite{greene2017jatisNIRCam}
%\nocite{schlawin2017tserSpeXPipeline}
\nocite{greene2016slitlessGrisms}
\nocite{schlawin2016kic1255}
\nocite{santerne2016ogle2011BLG-0417}
\nocite{stevenson2016ers}
\nocite{schlawin2010}
\nocite{schlawin2016kic1255}
\nocite{schlawin2014TSpec}
\nocite{muirheadKOI961}
\nocite{dale2009spitzerAnthology}
\nocite{west2011sloanMdwarf}
\nocite{muirhead2012}
\nocite{johnson2012}
\nocite{muirheadKOI961}
\nocite{muirhead2014coolKOIIV}

\begingroup
\renewcommand{\section}[2]{}%
\bibliographystyle{es_apj}
\bibliography{this_biblio}
\endgroup

%\vspace{-0.15in}
%\subsection*{Additional Products}
%\nocite{schlawin2014,schlawin2017dhs,schlawin2017bdVar,greene2016slitlessGrisms,Line2013}
%\begingroup
%\renewcommand{\section}[2]{}%
%\bibliographystyle{ieeetr}
%\bibliography{this_biblio}
%\endgroup

\vspace{-0.15in}
\subsection*{Awards}
\begin{itemize}[noitemsep]
        \item 2012-2013: Space Grant Fellowship Cornell
        \item 2009-2010: Space Grant Fellowship Cornell
        \item 2009, 2010: Honorable Mention NSF Graduate Research Fellowship 
        \item 2009: Oberlin College Norm Craig Student/Athlete Award
\end{itemize}

\subsection*{Postdoctoral Research Experience}
\begin{itemize}[noitemsep]
	\item 2015-2018: NIRCam GTO Transiting Exoplanet Program with JWST, led by Tom Greene. Simulated spectra with CHIMERA models, retrievals of atmospheric parameters, prepared observations with Astronomers Proposal Tool
	\item 2015-2018: NIRCam Support, PI Marcia Rieke. Mission support for NIRCam during cryo-vacuum testing at NASA Goddard, NASA Johnson and coordination with Space Telescope Science Institute's software and operations, including a proposed new observing mode
	\item 2015-2018: Brown Dwarf Rotational Modulation Observations in coordination with Adam Burgasser. Leading observing strategies and data analysis of high-precision, ground-based infrared observations.
\end{itemize}

\subsection*{Graduate Research Experience}
\begin{itemize}[noitemsep]
	\item 2011-2015: Cornell Exoplanet Spectra, advised by Terry Herter at Cornell University -- Analyzed infrared spectra of hot Jupiter exoplanets and the possibly disintegrating planet KIC 12557548b
	\item 2012-2015: TripleSpec4 Construction, PI Terry Herter at Cornell University -- Designed slit viewing optics, toleranced optics, constructed small parts and commissioned TS4, an infrared spectrograph for the Blanco telescope in Chile
	\item 2009-2015: Cornell Exoplanet Research Group, advised by James P. Lloyd at Cornell University -- Transit Spectroscopy -- Analyzed Hubble COS, ACS and STIS data to study the thermospheres of HD209458b and HD189733b taking into account limb brightening. Also observed at Palomar Observatory with the TEDI and TripleSpec instruments
\end{itemize}

\subsection*{Pre-Graduate Experience}
\begin{itemize}[noitemsep]
	\item 2008-2009: Herschel Inner Galaxy Gas Survey (HIGGS), advised by Chris Martin at Oberlin College. Honors Research -- Modeled radiative transfer of sub-millimeter lines to prepare for the Herschel Space Observatory
	\item 2008 Summer: Cooling Lines in the Interstellar Medium (ISM), advised by Danny Dale at the University of Wyoming -- Compared Mid-infrared and far-infrared data to find correlations in ISM coolants
	\item 2007-2008: Oberlin Pulsar Lab, advised by Dan Stinebring at Oberlin College --Used a 64 node supercomputer to simulate plasma lenses in the ISM and their effects on Secondary Spectra of pulsar light
	\item 2007 Summer: Wyoming Infrared Observatory REU, advised by Danny Dale at the University of Wyoming: Researched ISM cooling lines, comparing Spitzer mid-infrared data to Infrared Space Observatory far-infrared data with Danny Dale, University of Wyoming
	\item 2006 Summer: Oberlin Observatory CCD Imaging, advised by Chris Martin at Oberlin College -- Wrote a customized manual on CCD imaging, from capturing, to processing and analyzing.
	\item 2005 and 2004 Summers: Dusty Plasma Experiment, advised by Andrew Zwicker at Princeton Plasma Physics Lab (PPPL) -- studied the behavior of silicon dust clouds suspended in plasmas
\end{itemize}

\subsection*{Observing Proposals}
\begin{itemize}[noitemsep]
    \item 2017 Fall: 1 Night -- Continued Characterization of solar analogs for NIRCam flux calibration.
    \item 2016 Spring and Fall: 2 Nights -- Characterization of solar analogs for NIRCam flux calibration.
    \item 2016 Spring: 5 Nights -- Kuiper 61-inch Long Term Monitoring of the disintegrating planet candidate KIC 12557548 b.
    \item 2014 Fall: 5 Nights -- IRTF Observing Proposal: Addressing Discrepancies in KIC 12557548b?s Transmission Spectrum
    \item 2013 Fall: 4 Nights -- IRTF Observing Proposal -- A Transmission Spectrum of the Possibly Disintegrating Planet KIC 12557548b
    \item 2012 Fall: 1 Night - NOAO Observing Proposal -- Imager Corrected NIR Spectro-Photometry of a Hot Jupiter
\end{itemize}

\subsection*{Outreach and Mentoring Activities}
\begin{itemize}[noitemsep]
    \item 2018: Mentoring and Education in Science for Tucson (MESCIT), Tucson, AZ -- co-organizer. MESCIT is a math tutoring program that matches and supports University of Arizona Undergraduate tutors with high school students at the Ha:san Preparatory Leadership school, a bicultural public high school serving the Tohono O'odham Native students.
    \item 2016: Blue Marble Astrobiology Academy, Tucson, AZ -- planet activities teacher. Blue marble camp engages traditionally underserved school students in science engineering and math skills over the fun topic of planets, habitability and astrobiology.
    \item 2009-2015: Fuertes Observatory, Ithaca NY -- volunteer for open house telescope viewings and sky tours. The open houses offer members of the community and students a way to experience the sky from their backyard.
    \item 2012-2015: Kids' Science Day at the Big Red Barn, Ithaca, NY -- Head Organizer. In coordination with the Graduate Women in Science, KSD offers a way for kids to meet real scientists and break stereotypes of what a scientist should be.
    \item 2011-2012: Focus for Teens, Ithaca, NY -- Rocket Launch Teacher. Offers teenage students a fun way to learn about the Universe and get involved with hands on activities.
    \item 2010-2011: Expand Your Horizons Program for Middle School Girls in STEM, Ithaca, NY - Rocket Launch Teacher. An annual campus visit to Cornell University where participants get to choose areas in STEM that interest to them.
    \item 2009-2015: ``Curious? Ask an Astronomer'', Ithaca, NY - volunteer for website, live Q\&A and 2011 podcast, This Year in Exoplanets. ``Curious?" answers the publics' many different questions about space, the Solar System and the latest astronomy news.
    \item 2006-2008: Oberlin Astronomy Club, Oberlin OH - president. Brought membership from 0 to 250
 \end{itemize}

\subsection*{Service and Mentoring}
\begin{itemize}[noitemsep]
    \item 2015-2018: Undergraduate advising, University of Arizona - mentor for research project using cryo-vacuum laboratory data with NIRCam to optimize pipeline methods.
    \item 2015-2018: Graduate Mentoring Program, University of Arizona - graduate mentor
    \item 2014-2015: Colloquium committee, Cornell University - graduate representative
    \item 2011-2013: Graduate Women in Science (coed), Cornell University - Outreach Coordinator
    \item2010-2012: Astronomy Grads Network (AGN) organization, Cornell University - Treasurer
\end{itemize}


%\subsection{}



\end{document}  
