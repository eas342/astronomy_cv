\documentclass[11pt, oneside]{article}   	% use "amsart" instead of "article" for AMSLaTeX format
\usepackage[margin=1in]{geometry}                		% See geometry.pdf to learn the layout options. There are lots.
\geometry{letterpaper}                   		% ... or a4paper or a5paper or ... 
%\geometry{landscape}                		% Activate for rotated page geometry
%\usepackage[parfill]{parskip}    		% Activate to begin paragraphs with an empty line rather than an indent
\usepackage{graphicx}				% Use pdf, png, jpg, or eps§ with pdflatex; use eps in DVI mode
								% TeX will automatically convert eps --> pdf in pdflatex		
\usepackage{amssymb}
\usepackage{booktabs}
\usepackage[colorlinks=true,urlcolor=blue]{hyperref}
\usepackage{enumitem}
%\usepackage{bibunits}
\usepackage[numbers]{natbib}

%\defaultbibliography{this_biblio}
%\defaultbibliographystyle{ieetr}

\setlength{\bibsep}{0pt plus 0.3ex}

\newcommand{\apj}{ApJ}
\newcommand{\apjl}{ApJL}
\newcommand{\procspie}{SPIE}
\newcommand{\pasp}{PASP}
\newcommand{\rsquo}{'}

%SetFonts

%SetFonts


\title{Everett Schlawin}


\date{}							% Activate to display a given date or no date

\begin{document}
\maketitle
\vspace{-0.85in}
\begin{centering}
\rule{6in}{0.03in}
Postdoctoral Researcher

268 Steward Observatory\\
933 N Cherry Ave, Tucson AZ 85721, USA

github: \href{https://github.com/eas342}{eas342}

\end{centering}

\subsection*{Education}
\begin{itemize}[noitemsep]
	\item 2015: PhD, Cornell University in Ithaca, NY
		\begin{itemize}[noitemsep]
		\item Major: Astronomy, Minor: Earth and Atmospheric Sciences
		\item Thesis: Observations of Disintegrating, Evaporating and Hot Planet Atmospheres with Transmission Spectra
		\item Advisor: Professor Terry Herter
		\end{itemize}
	\item 2012: Masters, Cornell University in Ithaca, NY
		\begin{itemize}[noitemsep]
		\item Major: Astronomy, Minor: Earth and Atmospheric Sciences
		\end{itemize}
	\item 2009: Bachelors of Arts, Oberlin College in Oberlin, OH
		\begin{itemize}[noitemsep]
		\item Major: Physics w/ Astrophysics concentration
		\item Phi Beta Kappa, High Honors
		\end{itemize}
\end{itemize}

\vspace{-0.2in}
\subsection*{Publications}
\nocite{schlawin2014,schlawin2017dhs,schlawin2017bdVar,greene2017jatisNIRCam,schlawin2017tserSpeXPipeline,greene2016slitlessGrisms,schlawin2016kic1255,schlawin2010,schlawin2016kic1255,schlawin2014TSpec,muirheadKOI961}
\begingroup
\renewcommand{\section}[2]{}%
\bibliographystyle{apj}
\bibliography{this_biblio}
\endgroup

%\vspace{-0.15in}
%\subsection*{Additional Products}
%\nocite{schlawin2014,schlawin2017dhs,schlawin2017bdVar,greene2016slitlessGrisms,Line2013}
%\begingroup
%\renewcommand{\section}[2]{}%
%\bibliographystyle{ieeetr}
%\bibliography{this_biblio}
%\endgroup

\vspace{-0.15in}
\subsection*{Awards}
\begin{itemize}[noitemsep]
        \item 2012-2013: Space Grant Fellowship Cornell
        \item 2009-2010: Space Grant Fellowship Cornell
        \item 2009, 2010: Honorable Mention NSF Graduate Research Fellowship 
        \item 2009: Oberlin College Norm Craig Student/Athlete Award
\end{itemize}

\subsection*{Outreach and Mentoring Activities}
\begin{itemize}[noitemsep]
    \item 2016: Blue Marble Astrobiology Academy, Tucson, AZ -- planet activities teacher. Blue marble camp engages traditionally underserved school students in science engineering and math skills over the fun topic of planets, habitability and astrobiology.
    \item 2009-2015: Fuertes Observatory, Ithaca NY -- volunteer for open house telescope viewings and sky tours. The open houses offer members of the community and students a way to experience the sky from their backyard.
    \item 2012-2015: Kids' Science Day at the Big Red Barn, Ithaca, NY -- Head Organizer. In coordination with the Graduate Women in Science, KSD offers a way for kids to meet real scientists and break stereotypes of what a scientist should be.
    \item 2011-2012: Focus for Teens, Ithaca, NY -- Rocket Launch Teacher. Offers teenage students a fun way to learn about the Universe and get involved with hands on activities.
    \item 2010-2011: Expand Your Horizons Program for Middle School Girls in STEM, Ithaca, NY - Rocket Launch Teacher. An annual campus visit to Cornell University where participants get to choose areas in STEM that interest to them.
    \item 2009-2015: ``Curious? Ask an Astronomer'', Ithaca, NY - volunteer for website, live Q\&A and 2011 podcast, This Year in Exoplanets. ``Curious?" answers the publics' many different questions about space, the Solar System and the latest astronomy news.
    \item 2006-2008: Oberlin Astronomy Club, Oberlin OH - president. Brought membership from 0 to 250
 \end{itemize}

\subsection*{Observing Proposals}
2016 Spring and Fall ? Characterization of solar analogs for NIRCam flux calibration.
2016 Spring: 5 Nights ? Kuiper 61-inch Long Term Monitoring of the disintegrating planet candidate KIC 12557548 b.
2014 Fall: 5 Nights ? IRTF Observing Proposal: Addressing Discrepancies in KIC 12557548b?s Transmission Spectrum
2013 Fall: 4 Nights - IRTF Observing Proposal ? A Transmission Spectrum of the Possibly Disintegrating Planet KIC 12557548b
2012 Fall: 1 Night - NOAO Observing Proposal - Imager Corrected NIR Spectro-Photometry of a Hot Jupiter

\subsection*{Service and Mentoring}
\begin{itemize}[noitemsep]
    \item 2015-2017: Undergraduate advising, University of Arizona - mentor for research project using cryo-vacuum laboratory data with NIRCam to optimize pipeline methods.
    \item 2015-2016: Graduate Mentoring Program, University of Arizona - graduate mentor
    \item 2014-2015: Colloquium committee, Cornell University - graduate representative
    \item 2011-2013: Graduate Women in Science (coed), Cornell University - Outreach Coordinator
    \item2010-2012: Astronomy Grads Network (AGN) organization, Cornell University - Treasurer
\end{itemize}


%\subsection{}



\end{document}  